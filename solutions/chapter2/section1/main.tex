%-------------------------------------------------------------------------------
% chapter2/section1/main.tex

\begin{exercise}{Exercise 1.3 (a)}
    Let $\phi \colon \FSheaf \to \GSheaf$ be a morphism of sheaves on $X$.
    Show that $\phi$ is surjective if and only if 
    the following condition holds: for ever open set $\open \subseteq X$,
    and for every $s \in \GSheaf(\open)$, 
    there is a covering $\{\open_i\}$ of $\open$,
    and there are elements $t_i \in \FSheaf(\open_i)$
    such that $\phi(t_i) = s|_{U_i}$ for all $i$.
\end{exercise}
\begin{solution}
    Fix some open $\open \subseteq X$ 
    and some section $s \in \GSheaf(\open)$.
    If $\phi \colon \FSheaf \to \GSheaf$ is surjective,
    the image sheaf $\Image \phi$ is equal to $\GSheaf$,
    and we can consider $s$ as being in $\Image\phi(\open)$.
    Let $\HSheaf$ denote the presheaf image of $\phi$,
    so $\HSheaf^+$ is $\Image\phi$.
    Since $s \in \Image\phi(\open)$, 
    for any point $P_i \in \open$ we have an neighborhood $V_i$ 
    where $P_i \in V_i \subset \open$,
    and we have a section $s_i \in \HSheaf(V_i)$ such that
    for any point $Q \in V_i$ the germ $(s_i)_Q = s(Q)$.
    Since $\HSheaf$ is the presheaf image of $\phi$, 
    there is some $t_i \in \FSheaf(V_i)$ that maps to $s_i$,
    and we have $(\phi(t_i))_Q = s_i(Q) = s_i|_{V_i}(Q)$.
    More succinctly, $\phi(t_i) = s_i|_{V_i}$, 
    and so the collection of all $\{V_i\}_{P_i \in \open}$ that covers $U$ 
    and the corresponding sections $t_i \in \FSheaf(V_i)$ 
    are exactly what we were looking for.
    
    Conversely, we'll show that $\phi$ is surjective 
    by showing the induced map on stalks is surjective (Exercise 1.2).
    Take $P \in X$, and some $\langle \open, s \rangle \in \GSheaf_P$.
    For this open set $\open$ take $V$ to be the element
    of the open cover we are given for $\open$ that contains $P$. 
    Then there is some $t \in \FSheaf(V)$ such that $\phi(t) = s|_V$.
    Since $\langle V, s \rangle = \langle \open, s \rangle$,
    the germ of $t$ in $\FSheaf_P$ 
    will map to $\langle \open, s \rangle \in \GSheaf_P$.
\end{solution}

\begin{exercise}{Exercise 1.3 (b)}
    Give an example of a surjective morphism of sheaves 
    $\phi \colon \FSheaf \to \GSheaf$ and an open set $U$ such that
    $\phi(\open) \colon \FSheaf(\open) \to \GSheaf(\open)$ is not surjective.
\end{exercise}
\begin{solution}
    Let $\FSheaf$ and $\GSheaf$ each be the sheaf of holomorphic functions 
    on the punctured complex plane $\Comps^\ast$,
    and let $\phi$ be the map that sends $f \in \FSheaf(\open)$
    to $\exp(f) \in \GSheaf(\open)$.
    The induced map on global sections
    $\FSheaf(\Comps^\ast) \to \GSheaf(\Comps^\ast)$ 
    is not surjective: nothing maps to the identity.
    But any point $P \in \Comps^\ast$ is contained in a simply connected 
    open set where we can define a branch of the complex logarithm.
    Maps in $\Image \phi$ are defined 
    by their behavior at the stalk at each point,
    so we are able to define maps in $(\Image \phi)(\open)$
    that act like a complex logarithm on $\open$ 
    for any open set $\open$ by gluing together collections 
    of locally defined branches of the complex logarithm
    defined around each point $P \in \open$.
    The existence of these maps give us that $\Image \phi = \GSheaf$.
\end{solution}

\begin{exercise}{Exercise 1.18}
    \term{Adjoint Property of $\inv{f}$}. 
    Let $f \colon X \to Y$ be a continuous map of topological spaces.
    Show that for any sheaf $\FSheaf$ on $X$ 
    there is a natural map $\inv{f}f_\ast\FSheaf \to \FSheaf$,
    and for any sheaf $\GSheaf$ on $Y$
    there is a natural map $\GSheaf \to f_\ast\inv{f}\GSheaf$.
    Use these maps to show that there is a natural bijection of sets,
    for any sheaves $\FSheaf$ on $X$ and $\GSheaf$ on $Y$,
    \begin{equation*}
        \Hom_X(\inv{f}\GSheaf,\FSheaf) = \Hom_Y(\GSheaf,f_\ast\FSheaf)\,.
    \end{equation*}
    Hence we say that $\inv{f}$ is a \term{left adjoint} of $f_\ast$,
    and that $f_\ast$ is a \term{right adjoint} of $\inv{f}$.
\end{exercise}
\begin{solution}
    Let $\HSheaf$ denote the presheaf on $X$ under which
    $\open \mapsto \varinjlim_{V \supset f(\open)} f_\ast\FSheaf(V)$.
    So $\HSheaf$ the presheaf to which 
    $\inv{f}f_\ast\FSheaf$ is associated.
    Now $f_\ast\FSheaf(V)$ is just $\FSheaf(\inv{f}(V))$,
    so for each $\inv{f}(V) \supset \open$ 
    we get an induced restriction map 
    $\rho_V \colon \FSheaf(\inv{f}(V)) \to \FSheaf(\open)$.
    Since $\HSheaf(\open)$ is the colimit over all $\FSheaf(\inv{f}(V))$,
    and we have a morphism $\rho_V$ for each $\inv{f}(V) \supset \open$
    by the univeral property of colimits we get a unique map
    $\HSheaf(\open) \to \FSheaf(\open)$,
    which gives us a presheaf morphism $\HSheaf \to \FSheaf$.
    Then by sheafifying $\HSheaf$ we get a unique morphism of sheaves 
    $\inv{f}f_\ast\FSheaf \to \FSheaf$.

    ...
\end{solution}
